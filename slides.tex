\documentclass[aspectratio=1610]{beamer}

\usetheme{Hannover}
% \usecolortheme{crane}
\setbeamertemplate{navigation symbols}{}

\author{Matthias Hutzler}
\title[Synthetic Algebraic Geometry]{Introduction to\\Synthetic Algebraic Geometry}
\date{Proof and Computation 2023\\Herrsching}


\begin{document}

\begin{frame}
  \maketitle
\end{frame}

\section{Overview}

\begin{frame}
  \frametitle{What is synthetic algebraic geometry?}
  \onslide<1>{\begin{description}[synthetic algebraic geometry]
    \item[algebraic geometry]
      Thinking about algebra in geometric terms.
    \item[synthetic algebraic geometry]
      Thinking about algebra in geometric terms,\\
      using a \alert{special language} that makes complex constructions look simple.
  \end{description}}

  \onslide<2>{
  Thinking about algebra\\
  in geometric terms.

  (non-synthetic)\\
  algebraic geometry

  synthetic\\algebraic geometry}
\end{frame}

\begin{frame}
  \frametitle{Why \alert{you} might be interested}

  \begin{itemize}
    \item
      apply constructive reasoning
    \item
      apply homotopy type theory
    \item
      understand some algebraic geometry
  \end{itemize}
\end{frame}

\begin{frame}
  \frametitle{Understanding algebraic subtleties geometrically}

  $X^2 = 0$
\end{frame}

\begin{frame}
  \frametitle{Motivation for non-affine schemes}
\end{frame}

\begin{frame}
  \frametitle{\includegraphics[width=5cm]{./images/agda-logo.png}}

  Agda is
  \begin{itemize}
    \item
      a functional programming language
      \begin{itemize}
        \item
          influenced by Haskell
      \end{itemize}
    \item
      a proof assistant
      \begin{itemize}
        \item
          with a \alert{cubical type theory} mode
      \end{itemize}
  \end{itemize}
\end{frame}

\end{document}
